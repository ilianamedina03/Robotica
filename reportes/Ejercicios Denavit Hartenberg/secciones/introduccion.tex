\section{Ejercicios Denavit Hartenberg}

El presente documento contiene la resolución de cinco ejercicios  siguiendo las indicaciones establecidas en la tarea. Cada ejercicio ha sido desarrollado a mano y verificado utilizando MATLAB a través del archivo \texttt{test/prueba\_DH.mlx}.

Para representar la cinemática de los robots utilizados en los ejercicios, se emplea el método de Denavit-Hartenberg (DH), una herramienta fundamental en la robótica para describir la posición y orientación de eslabones y articulaciones de manera sistemática.

Para complementar la resolución, se han generado tablas en el formato robot1.csv, dependiendo del robot seleccionado. Todos los archivos han sidosubidos al repositorio de GitHub según lo solicitado.

Además, este informe incluye una sección dedicada a cada ejercicio, en la cual se presentan dos imágenes: una con el diagrama y flechas dibujadas manualmente, y otra correspondiente a la figura generada en MATLAB.
