\chapter{Introducción} \label{chap:introduccion}

El presente documento corresponde al reporte final de la materia de Robótica, en el cual se integran los conceptos y métodos abordados durante el semestre. El enfoque principal se centra en el análisis cinemático de robots manipuladores mediante el uso del método de Denavit-Hartenberg (DH), técnica ampliamente utilizada para describir la posición y orientación de eslabones y articulaciones de manera sistemática.

Durante el semestre, se diseñó y modeló un robot tipo manipulador utilizando SolidWorks, estableciendo sus parámetros geométricos y dinámicos. A partir de este modelo, se implementaron los algoritmos cinemáticos en MATLAB, lo que permitió generar simulaciones detalladas del comportamiento del efector final bajo distintas trayectorias. Adicionalmente, se emplearon herramientas como ROS, Gazebo y RViz para llevar a cabo la simulación en un entorno virtual, integrando aspectos de control, planificación de trayectorias y visualización en tiempo real.

El reporte incluye la resolución de ejercicios específicos aplicando la metodología DH, los cuales fueron desarrollados manualmente y comprobados computacionalmente. Cada ejercicio se acompaña de representaciones gráficas, tanto esquemáticas como generadas por software, que permiten visualizar la estructura y el movimiento del robot.

Este informe tiene como finalidad demostrar la aplicación práctica de los conocimientos adquiridos, consolidando la comprensión del modelado cinemático en el contexto del diseño y análisis de sistemas robóticos.


Este documento reúne el trabajo colaborativo del equipo, demostrando no solo el dominio de las bases de la robótica, sino también el aprendizaje de herramientas prácticas que fortalecen las habilidades en ingeniería aplicada. Las secciones que siguen presentan el marco teórico, desarrollo del proyecto, resultados obtenidos y conclusiones personales que reflejan el impacto del curso en la formación de cada integrante.