\chapter{Introducción} \label{chap:introduccion}

El presente documento corresponde al reporte final de la materia de Robótica, en el cual se integran los conceptos y métodos abordados durante el semestre. El enfoque principal se centra en el análisis cinemático de robots manipuladores mediante el uso del método de Denavit-Hartenberg (DH), técnica ampliamente utilizada para describir la posición y orientación de eslabones y articulaciones de manera sistemática.

Como parte del desarrollo, se modeló un robot utilizando la herramienta de diseño asistido por computadora SolidWorks, al cual se le aplicaron los parámetros DH para establecer su modelo cinemático. Posteriormente, se implementó dicho modelo en el entorno de MATLAB con el objetivo de verificar y simular su comportamiento.

El reporte incluye la resolución de ejercicios específicos aplicando la metodología DH, los cuales fueron desarrollados manualmente y comprobados computacionalmente. Cada ejercicio se acompaña de representaciones gráficas, tanto esquemáticas como generadas por software, que permiten visualizar la estructura y el movimiento del robot.

Este informe tiene como finalidad demostrar la aplicación práctica de los conocimientos adquiridos, consolidando la comprensión del modelado cinemático en el contexto del diseño y análisis de sistemas robóticos.

