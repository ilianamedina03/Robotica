\chapter{Marco Teórico} 
\label{chap:marco_teorico}

El análisis cinemático de robots constituye una de las bases teóricas más relevantes en el estudio de la robótica. A través de esta disciplina es posible describir el comportamiento geométrico de un robot, determinando su posición y orientación en el espacio en función de variables articulares. El presente marco teórico tiene como finalidad contextualizar los conceptos fundamentales que sustentan el trabajo desarrollado en este informe, abarcando temas como la robótica, la cinemática de manipuladores, el método de Denavit-Hartenberg, y el uso de herramientas computacionales como SolidWorks y MATLAB para el diseño, modelado y simulación.