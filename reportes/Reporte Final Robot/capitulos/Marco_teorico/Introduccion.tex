\chapter{Marco Teórico} 
\label{chap:marco_teorico}

El análisis cinemático de robots constituye una de las bases teóricas más importantes en el estudio de la robótica, ya que permite describir el comportamiento geométrico de un manipulador en función de sus variables articulares. Este capítulo tiene como finalidad contextualizar los conceptos clave que sustentan el trabajo realizado a lo largo del proyecto.

Se abordan temas como la cinemática directa, diferencial e inversa, empleando el método de Denavit-Hartenberg para modelar de manera sistemática la posición y orientación del robot. Además, se incluyen herramientas computacionales como SolidWorks y MATLAB, utilizadas para el diseño, simulación y validación de los modelos.

También se presenta una visión general del uso de ROS y sus componentes, así como principios básicos de dinámica y control, los cuales son esenciales para interpretar el comportamiento del robot y su respuesta ante diferentes trayectorias.