\section{Características del Robot} \label{sec:caracteristicas_del_robot}

\begin{table}[ht]
	\centering
	\caption{Parámetros de Denavit Hartenberg y límites del robot}
	\label{tab:parametros_robot}
	\begin{tabular}{l|ccccccccc}
		\toprule
		N & {$\theta$} & {$d$} & {$a$} & {$\alpha$} & {tipo} 
		& {$q_{\min}$} & {$q_{\max}$} 
		& {$\dot q_{\max}$} & {$\ddot q_{\max}$} \\
		\midrule
		1 & 0   & 496.00 & 0   & 90 & r & -180 & 180 & 180 & 360 \\
		2 & 90  & 0.00   & 500 & 0  & r & -90  & 90  & 180 & 360 \\
		3 & 90  & 451.38 & 0   & 90 & r & -180 & 0   & 180 & 360 \\
		4 & 90  & 195.00 & 0   & 90 & r & -180 & 0   & 180 & 360 \\
		\bottomrule
	\end{tabular}
\end{table}

\bigskip
\noindent
\textbf{Donde:}
\begin{description}
	\item[N] Número de la articulación.
	\item[\(\theta\)] Ángulo articular (grados).
	\item[\(d\)] Desplazamiento articular (unidades de longitud).
	\item[\(a\)] Longitud del eslabón (unidades de longitud).
	\item[\(\alpha\)] Ángulo de torsión DH (grados).
	\item[tipo] ‘r’ para articulación rotacional, ‘p’ para prismática.
	\item[\(q_{\min}\), \(q_{\max}\)] Límites de posición (grados o unidades de desplazamiento).
	\item[\(\dot q_{\max}\)] Límite de velocidad (grados/s o unidades/s).
	\item[\(\ddot q_{\max}\)] Límite de aceleración (grados/s² o unidades/s²).
%	\item[\(\tau\)] Torque o fuerza máxima permitido (\(N \cdot m\) o \(N\)).
%	\item[\(\mu_s\)] Fricción estática (\(N\) o \(N \cdot m\)).
%	\item[\(\mu_k\)] Fricción cinética (\(N \cdot m \cdot s\) o \(\frac{N \cdot m \cdot s}{rad}\)).
\end{description}


\subsection{Importancia de los Límites y Parámetros Dinámicos}

\begin{itemize}
	\item Los \textbf{límites articulares} aseguran que el robot opere dentro de sus capacidades físicas, evitando colisiones o daños mecánicos.
	\item Las velocidades máximas ($\dot{q}_{\text{max}}$) y aceleraciones máximas ($\ddot{q}_{\text{max}}$) son fundamentales para realizar simulaciones realistas y seguras.
	\item Estos parámetros también se utilizan como \textbf{restricciones} durante el planeamiento de trayectorias, para asegurar que los movimientos generados sean físicamente viables y eficientes.
\end{itemize}

