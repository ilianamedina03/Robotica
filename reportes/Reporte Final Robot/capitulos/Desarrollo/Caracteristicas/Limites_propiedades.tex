\subsection{Límites y propiedades dinámicas de las articulaciones} \label{subsec:limites_propiedades}

%Básicamente, explicarán lo que aparece en la \autoref{tab:parametros_robot}: Parámetros de Denavit Hartenberg y límites del robot, específicamente lo que aparece después de \code{tipo}. Como no completamos la tarea de dinámica,. pueden comentar esta subsección del documento principal.
El modelo de Denavit-Hartenberg (DH) sirve como base para calcular las matrices de transformación homogénea entre cada par de eslabones. Estas matrices permiten obtener la posición y orientación del efector final a lo largo del tiempo mediante la \textbf{cinemática directa}.

\begin{itemize}
	\item La posición cartesiana $(X, Y, Z)$ y la orientación $(\phi, \theta, \psi)$ del efector se calculan para cada instante de la trayectoria simulada.
	\item Las \textbf{velocidades lineales} y \textbf{aceleraciones lineales} se obtienen derivando numéricamente las posiciones.
	\item Las \textbf{velocidades angulares} y \textbf{aceleraciones angulares} se calculan a partir de los cambios en la orientación, por ejemplo, derivando los ángulos de Euler.
\end{itemize}

\vspace{0.5em}