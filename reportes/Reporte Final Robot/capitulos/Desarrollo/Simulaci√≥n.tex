\section{Simulación} \label{sec:simulacion}
En este proyecto se realizó la simulación de un robot soldador llamado Assembly utilizando ROS, Gazebo y MoveIt, con desarrollo en Visual Studio Code. Assembly, a diferencia de otros brazos robóticos, no cuenta con pinzas ni electroimán, ya que su función principal es ejecutar movimientos precisos orientados a tareas de soldadura. Para ello, se creó un archivo URDF que define su estructura mecánica y propiedades dinámicas, utilizado tanto para la visualización como para la planificación de movimientos. Con el asistente de configuración de MoveIt se generaron los paquetes necesarios para la simulación y control del robot, permitiéndole ejecutar movimientos descendentes que simulan el proceso de soldadura. Además, se realizaron pruebas en las que el robot operó a torque máximo y mínimo, evaluando su comportamiento bajo distintas condiciones de esfuerzo. Todo el entorno fue montado en Ubuntu 20.04, el cual fue ejecutado mediante VirtualBox para facilitar la compatibilidad y portabilidad del sistema.