\section{Iliana}
La materia de Robótica fue, para mí, un camino lleno de retos técnicos y personales. Desde el inicio, me enfrenté a conceptos complejos y herramientas con las que no estaba familiarizada, como utilizar LATEX, lo que generó momentos de frustración. Sin embargo, cada obstáculo se convirtió en una oportunidad para aprender y crecer. El proyecto final fue particularmente desafiante y una experiencia muy completa, ya que implicó enfrentar errores, replantear estrategias y colaborar y apoyarme constantemente con mis compañeros. Fue estresante, sobre todo cuando algo no salía como esperábamos, pero también hubo muchos aprendizajes. Aprendí a tener más paciencia, a confiar en el proceso y a no rendirme cuando algo no funcionaba a la primera, aunque fue  gratificante ver los resultados concretos de nuestro trabajo.Gracias a esta materia, no solo adquirí conocimientos técnicos, sino que también fortalecí mi paciencia, mi capacidad de análisis y mi actitud ante los problemas. Robótica me enseñó que equivocarse también es parte del proceso y que con esfuerzo y dedicación, todo aprendizaje es posible.

