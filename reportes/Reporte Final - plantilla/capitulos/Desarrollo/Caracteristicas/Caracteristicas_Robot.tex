\section{Características del Robot} \label{sec:caracteristicas_del_robot}

En esta parte deben de explicar en una tabla la información principal del robot. Para ello, pueden cambiar el archivo de excel \ffile{Robot\_Resumen.xlsx} que está en \ffolder{tablas} y pedirle a ChatGPT que lo convierta a \LaTeX o simplemente pasarlo a un convertidor en línea.

\begin{table}[ht]
	\centering
	\caption{Parámetros de Denavit Hartenberg y límites del robot}
	\label{tab:parametros_robot}
	\begin{tabular}{ l|cccccccccccc
		}
		\toprule
		N & {$\theta$} & {$d$} & {$a$} & {$\alpha$} & {tipo} 
		& {$q_{\min}$} & {$q_{\max}$} 
		& {$\dot q_{\max}$} & {$\ddot q_{\max}$} 
		& {$\tau$} & {$\mu_s$} & {$\mu_k$} \\
		\midrule
		1 & 0 & 7 & 3  & 0   & r & -90 & 90 & 180 & 360 &  8  & 0.1 & 0.2 \\
		2 & 0 & 0 & 3  & -90 & r & -90 & 90 & 180 & 360 & 50  & 0.1 & 0.2 \\
		3 & 0 & 2 & 0  & -90 & r & -90 & 90 & 180 & 360 & 30  & 0.1 & 0.2 \\
		4 & 0 & 5 & 0  & 0   & p &   3 &  7 &   1 &   2 & 2  & 0.1 & 0.2 \\
		\bottomrule
	\end{tabular}
\end{table}
\bigskip
\noindent
\textbf{Donde (cambien las unidades):}
\begin{description}
	\item[N] Número de la articulación.
	\item[\(\theta\)] Ángulo articular (grados).
	\item[\(d\)] Desplazamiento articular (unidades de longitud).
	\item[\(a\)] Longitud del eslabón (unidades de longitud).
	\item[\(\alpha\)] Ángulo de torsión DH (grados).
	\item[tipo] ‘r’ para articulación rotacional, ‘p’ para prismática.
	\item[\(q_{\min}\), \(q_{\max}\)] Límites de posición (grados o unidades de desplazamiento).
	\item[\(\dot q_{\max}\)] Límite de velocidad (grados/s o unidades/s).
	\item[\(\ddot q_{\max}\)] Límite de aceleración (grados/s² o unidades/s²).
	\item[\(\tau\)] Torque o fuerza máxima permitido (\(N \cdot m\) o \(N\)).
	\item[\(\mu_s\)] Fricción estática (\(N\) o \(N \cdot m\)).
	\item[\(\mu_k\)] Fricción cinética (\(N \cdot m \cdot s\) o \(\frac{N \cdot m \cdot s}{rad}\)).
\end{description}

