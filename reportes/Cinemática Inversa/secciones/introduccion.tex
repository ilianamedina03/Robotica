\section{Ejercicios Denavit Hartenberg}

El presente documento contiene el desarrollo de ejercicios enfocados en la cinemática inversa de dos robots diferentes al robot 1, conforme a las indicaciones establecidas en la tarea. Cada robot fue programado y simulado utilizando MATLAB, generando animaciones que muestran el movimiento del robot al alcanzar un objetivo específico en el espacio.

La cinemática inversa permite determinar los ángulos articulares necesarios para que el efector final del robot alcance una posición deseada. En este reporte se presentan los resultados obtenidos para cada uno de los robots seleccionados, así como las gráficas correspondientes a su trayectoria.

Además, se calcula y analiza el error del objetivo alcanzado, con el fin de evaluar la precisión de cada solución obtenida. Los videos generados durante la simulación han sido guardados y, junto con los archivos de código, pueden encontrarse en el repositorio correspondiente o se adjuntan a este reporte según lo solicitado.