\section{\textbf{Conclusión}} Los sensores son herramientas esenciales para que los robots y otros dispositivos puedan percibir y reaccionar a su entorno. En este reporte, se exploraron distintos tipos de sensores, tanto internos como externos, que ayudan a medir posición, velocidad, aceleración y fuerza. También se analizaron sensores más avanzados, como los giroscopios, acelerómetros, magnetómetros y el sistema LiDAR.

Cada sensor tiene una función importante. Los sensores de movimiento aseguran que los robots se desplacen con precisión, mientras que los de aceleración y fuerza ayudan a mejorar la estabilidad y seguridad. Por otro lado, sensores como el giroscopio, el magnetómetro y el LiDAR permiten a los robots y vehículos autónomos ubicarse y moverse en su entorno con mayor exactitud.

Gracias a estos avances, los sensores se han convertido en una parte clave de muchas industrias, como la robótica, la exploración espacial, la seguridad y el transporte. Su continuo desarrollo permite mejorar la tecnología y facilitar tareas en nuestra vida diaria. En el futuro, seguirán evolucionando y ayudando a crear sistemas más inteligentes y eficientes.